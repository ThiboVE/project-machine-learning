\documentclass[twoside,11pt]{article}

\usepackage{blindtext}
\usepackage{jmlr2e}
\usepackage[style=apa]{biblatex}
\addbibresource{References.bib}
% If you need any other packages, you can add them here.

\usepackage[version=4]{mhchem} %To write nice chemical equations
\usepackage{placeins}%
\usepackage{mathtools}
\usepackage[utf8]{inputenc}
\usepackage[margin=2.5cm]{geometry} %Makes the margin smaller
\usepackage[labelfont=bf, skip=5pt, font=small]{caption} %Needed for nice caption of tables and figures
\usepackage[labelfont=bf, skip=5pt, font=small]{subcaption} %Needed for nice subcaption of tables and figures
\usepackage[export]{adjustbox} %Needed for proper figure adjustment
\usepackage[table,xcdraw]{xcolor}
\usepackage{hyperref} %To put hyperlinks in the document
\usepackage{pgfplots} %To import pgf files/pictures
\usepackage{multirow}
\usepackage{amsmath}
\usepackage{bookmark}
\usepackage{booktabs}
\usepackage{pgf}
\usepackage{float}
\usepackage{xcolor}
\usepackage{graphicx} % Required for inserting images
\usepackage{wrapfig}
% \usepackage[numbers, super, sort&compress]{natbib}
\usepackage[capitalise]{cleveref}
\usepackage[table,xcdraw]{xcolor} % Voor kleurgebruik in tabelle
\usepackage{siunitx}
\usepackage{url}
\usepackage{pdfpages}
\usepackage{braket}
\usepackage[most]{tcolorbox}
\usepackage{lipsum} % For sample text

% Here you can add any macros you might need.
\newcommand{\dataset}{{\cal D}}
\newcommand{\fracpartial}[2]{\frac{\partial #1}{\partial  #2}}

\usepackage{lastpage}

\crefname{figure}{Figure}{Figures}
\Crefname{figure}{Figure}{Figures}
\crefname{subfigure}{Figure}{Figures}
\Crefname{subfigure}{Figure}{Figures}
\crefname{section}{Section}{Sections}
\Crefname{section}{Section}{Sections}
\crefname{equation}{Equation}{Equations}
\Crefname{equation}{Equation}{Equations}
\crefname{table}{Table}{Tables}

\setlength{\parindent}{0em} %Causes there not to be indentations when creating a new paragraph.

% \numberwithin{equation}{section}
% \numberwithin{table}{section}
% \numberwithin{figure}{section}
\hypersetup{
    colorlinks=true,
    linkcolor=blue,
    filecolor=magenta,      
    urlcolor=cyan,
    pdftitle={play},
    citecolor=black,
    }
% \bibliographystyle{achemso} 

\firstpageno{1}
\begin{document}

\title{Machine learning for HOMO-LUMO-gap Prediction and Inverse Molecular Design}

\author{\name Mattice Criel
  \AND
  \name Yarno De Jaeger
  \AND
  \name Thibo Van Eeckhoorn}

\maketitle

\begin{abstract}
  Accurate predictions of molecular electronic properties such as the HOMO–LUMO gap and inverse design of molecules based on such properties are critical for materials discovery and molecular engineering, but remain costly and time consuming with experimental methods and computationally expensive with quantum-chemical methods. While machine learning has gained increased relevance in such applications as a less expensive alternative, modern models still fail to reliably produce chemically viable results. In this study, we benchmark a LightGBM model against a GCN model for property prediction and a VAE-based model for molecular design. Our results show that the LightGBM and GCN model achieve comparable HOMO-LUMO gap prediction performance, while the VAE model shows good reconstruction capabilities, but limited chemical validity and property control for the generation of new molecules.
  \textcolor{purple}{Only need to add one sentence for \textbf{Outlook}Rewrite abstract to only contain 5 sentences:}
  \textbf{Context:}
  \textbf{Problem:}
  \textbf{Solution:}
  \textbf{Results:}
  \textbf{Outlook:}
\end{abstract}

\section{Introduction}

% In the introduction, give a brief overview of the field your research is in and the problem you are trying to solve or the question you are trying to answer. Then briefly explain what you did and what your conclusions are. This can already be done in a little more detail than in the abstract, but the fully exact description of your methods is for the ``Methods'' section. You can think of this section as an explanation of your research to someone who has 5 to 10 minutes to spare. Below are two more paragraphs of text. As you can see, the first line of a new paragraph is automatically indented. You don't need to change anything here.

% Within the spectrum of discrete energy levels (molecular orbitals) in a molecule that can be filled with electrons, the HOMO-lUMO gap is the energy difference between the highest occupied molecular orbital (HOMO) and lowest unoccupied molecular orbital (LUMO). This is a fundamental descriptor of the electronic structure of a molecule and is strongly influenced by the specific molecular structure and its functional groups. The HOMO-LUMO gap governs key properties such as reactivity, optical absorption and charge transport characteristics and is directly related to the band-gap in conductivity, determining whether a material behaves as a conductor, insulator or semi-conductor\parencite{dwivedi_2025}. The HOMO-LUMO gap also influences the effiency of organic photovoltaics (solar cells based on organic molecules) and organic light-emitting devices (OLED-technology) \parencite{liu_2015}. As a consequence, accurate determination of the HOMO-LUMO gap and design of molecular materials with a specified HOMO-LUMO gap is essential for both materials science and molecular engineering.

The HOMO-LUMO gap is the energy difference between the highest occupied and lowest unoccupied molecular orbitals and is a fundamental descriptor of a molecule's electronic structure. It strongly depends on molecular structure and functional groups and governs key properties such as chemical reactivity, optical absorption, charge transport, and conductivity, determining whether a material behaves as a conductor, semiconductor, or insulator \parencite{dwivedi_2025}. The HOMO-LUMO gap is also critical for the performance of organic photovoltaics (OPVs) and organic light-emitting devices (OLED) \parencite{liu_2015}, making its accurate prediction and targeted design central to molecular engineering and materials science.
\newline
\newline
\noindent
Experimentally, HOMO-LUMO gaps are typically estimated using optical spectroscopy or voltammetry\parencite{Sworakowski2018,COSTA201651}. However, these measurements require synthesized, purified materials making large-scale exploration of hypothetical chemical structures costly and time consuming. Computational quantum chemistry methods such as Hartree-Fock and Density functional theory offer accurate alternatives but are computationally expensive and scale poorly with molecular size and dataset volume, limiting their applicability for large-scale molecular exploration.
\newline
\newline
\noindent
Machine learning provides an efficient alternative by enabling rapid prediction of electronic properties\parencite{Hasan2025}. Crucially, large-scale quantum-chemical datasets like QM9, contain extensive information about molecular structures and properties of organic molecules, enabling the development and benchmarking of ML models with the data volume required for robust learning. Beyond property prediction, machine learning also enables generative molecular design, allowing in silico exploration and optimization of hypothetical molecules, thereby reducing experimental cost and accelerating molecular discovery.
\newline
\newline
\noindent
In this work, we develop a unified framework for HOMO-LUMO gap prediction and inverse molecular design. We first construct predictive models using a descriptor-based LightGBM approach and a graph convolutional network (GCN) that learns directly from molecular graph topology, achieving comparable accuracy and demonstrating the complementary strengths of traditional ensemble methods and graph-based deep learning. Building on the GCN, we further explore conditional molecular generation by integrating it into a conditional variational autoencoder (cVAE) to generate molecules targeting specified HOMO-LUMO gaps. While the predictive models demonstrate strong performance and generalization, the generative results highlight persistent challenges in achieving reliable property control and valid molecular generation, underscoring the need for improved training strategies, more expressive latent representations, and more effective decoding mechanisms.

\section{Related work}
% In the related work section, you briefly describe which other papers have performed research on the same topic as you. Try to explain what the differences are between your research and theirs. You can cite papers like this: \parencite{chow:68}. If you use the citation
% in the middle of a sentence, you can do it like this: \textcite{chow:68} showed that \dots
\noindent
In recent years, significant progress has been made in predicting HOMO-LUMO gaps using various machine learning techniques, including deep learning approaches, ensemble methods, and graph-based models. Early descriptor-based regression models achieved mean absolute errors (MAEs) in the range of approximately 0.10-0.25 eV using methods such as Random Forests and LightGBM trained on molecular connectivity and handcrafted descriptor features \parencite{Goh2022LGBStack}. One notable example is the study by \textcite{Hasan2025}, who used a database of molecular properties computed with Hartree–Fock theory and reduced the MAE of a LightGBM model to 0.1675 eV.
\newline
\newline
\noindent
In this context, LightGBM is adopted in the present work due to its proven effectiveness for molecular property prediction using descriptor-based representations. As a gradient boosting decision tree algorithm, LightGBM is well suited to modeling complex, non-linear relationships between molecular descriptors and electronic properties while maintaining strong generalization performance on medium-sized datasets. Moreover, LightGBM offers advantages in terms of computational efficiency, robustness to feature scaling, and interpretability through feature importance analysis. These characteristics make it a strong and well-established baseline, enabling fair comparison with prior descriptor-based approaches while providing competitive accuracy without the added complexity of deep or graph-based models.
\newline
\newline
\noindent
Graph neural networks (GNNs) have emerged as a powerful tool to predict molecular properties from graph representations of molecular structures. As  there is still a lack of 3D graph-based and 3D grid-based methods for molecular property prediction\parencite{LI2022103373}, we will focus on a 2D graph-based model. One such model that is common for molecular property predicted, but also specifically for HOMO-LUMO prediction, is the graph convolutional network (GCN)\parencite{NEURIPS2023_cc83e973,choi2022scalabletraininggraphconvolutional,Therrien2025}.
\newline
\newline
\noindent
\textcite{NEURIPS2023_cc83e973} demonstrated that deep-learning models generally are unable to outperform non-deep ones for molecular property prediction. However, \textcite{NEURIPS2023_cc83e973} also report MAEs of the predicted HOMO-LUMO gap (using the QM9 dataset) that indicate graph-based deep-learning models as more accurate than traditional models. To more clearly assess whether graph-based deep-learning models outperform traditional models for HOMO–LUMO gap prediction, we chose to compare the performance of a LightGBM model with that of a graph convolutional network (GCN).
\newline
\newline
\noindent
In the context of generative models, variational autoencoders (VAEs) have emerged as a standard framework for de novo molecular design\parencite{walters_2021}. One of the more recent influential works is done by \textcite{bombarelli_2018} applied VAEs to construct a continuous and differentiable latent space from SMILES representations of molecules, enabling gradient-based optimization of molecular properties using a Gaussian process. This approach has inspired numerous subsequent studies \parencite{blaschke_2018,Yoshikai2024,kusner_2017}. Notably, \textcite{lim_2018} introduced a conditional VAE framework for property-controlled molecular generation, demonstrating the advantages of incorporating molecular property information into the encoding process and manipulating it during decoding. However, many studies have highlighted limitations of SMILES-based representations, suggesting that graph-based encoders and decoders better capture molecular structure and improve the validity of generated molecules. These findings motivate the graph-based generative approach adopted in our work.

\section{Methodology}
% In this section, explain in detail what you did. Mention exactly what techniques you used, how you implemented everything, what experiments you performed, \dots This section should contain enough detail so that someone else can completely repeat your research.

The source code for this project is available on Github\parencite{ThiboVE_project_machine_learning}.

\subsection{Dataset}
For all models in this study, we used the QM9 dataset\parencite{PyG_QM9}, a widely adopted quantum chemistry dataset comprising 130,831 small organic molecules with associated properties, including the HOMO--LUMO gap. Only molecules with valid SMILES representations were retained, leaving 129,012 molecules, can be seen in \cref{fig:gap_distribution}. SMILES (Simplified Molecular Input Line Entry System) is a text-based notation that encodes chemical structures as ASCII strings, enabling efficient computational processing; for example, benzene is represented as \texttt{C1=CC=CC=C1}. While SMILES strings can be interpreted by computers, they are not directly suitable as input for most machine learning models, which require numerical representations. Consequently, preprocessing is necessary to transform molecular structures into informative, fixed-length numerical features suitable for model training.
\begin{figure}[H]
    \centering
    \includegraphics[width=0.4\textwidth]{img/bandgap_distribution_dataset.png}
    \caption{Illustration of the bandgap distribution in the QM9 dataset where only the valid molecules are kept. The vertical dashed lines indicates the minimum and maximum bandgap values in the dataset.}
    \label{fig:gap_distribution}
\end{figure}
\subsection{Molecular Representations and Preprocessing}
In this study, we focus on two different models for HOMO-LUMO gap prediction: LightGBM, a classical machine learning model and GCN, a deep learning model, which require different types of molecular representations. Therefore, two types of molecular representations were generated from the SMILES strings.

\subsubsection{Descriptor-based features for LightGBM}
Using RDKit, a set of 217 molecular descriptors was generated for each molecule from the SMILES representation\parencite{rdkit}. These descriptors capture various aspects of molecular structure and properties, including molecular size, shape, electronic characteristics, surface area, and the presence of functional groups. Together, they provide a fixed-length numerical representation of each molecule suitable for machine learning models such as LightGBM.
    
\subsubsection{Graph-based representation for GCN and cVAE:}  
For graph neural networks, molecules are represented as graphs where atoms are nodes and bonds are edges. As the QM9 dataset represents molecules as SMILES, all molecule SMILES were translated to a node and edge matrix using the $Chem$ package of RDKit\parencite{rdkit_chem_docs}. Node features include atomic number, formal charge, hybridization, aromaticity, and whether the atom is in a ring structure. The edge matrix was defined as an adjacency matrix where the diagonal elements were set to 1, indicating a self-connection, which makes the matrix amenable to convolutions\parencite{deshmukh2023building}. The distance of each bond was included in the edge matrix as 1 over the bond distance. The representation of a molecule as a node and edge matrix allows the GCN to learn structural features directly from molecular graphs without relying on precomputed descriptors.

\subsection{Data Splitting}
To ensure unbiased evaluation, the dataset was split into training, validation, and test sets. For all models, nested cross-validation was used:

\begin{itemize}
  \item \textbf{Outer loop:} 5-fold cross-validation to estimate generalization performance
  \item \textbf{Inner loop:} 10-fold cross-validation for hyperparameter tuning
\end{itemize}

The splits were stratified based on the distribution of the HOMO-LUMO gap values into 10 quantile bins to preserve this distribution across the different folds. 

\subsection{Evaluation metrics}\label{sec:evalmetrics}

Since both the LightGBM and GCN models are used to predict HOMO-LUMO gaps and are directly compared, the same evaluation metrics are applied to both. Commonly used metrics for HOMO-LUMO gap prediction include the mean absolute error (MAE, \cref{eq:MAE}), root mean squared error (RMSE, \cref{eq:RMSE}), and the coefficient of determination ($R^2$, \cref{eq:R2}) between the predicted and true HOMO-LUMO gaps.

The cVAE model performance is evaluated using the average character-level reconstruction accuracy ($\overline{CRA}$, \cref{eq:CRA}), the chemical validity of newly generated SMILES strings using RDKit (\cref{eq:validity}), and the agreement between the target HOMO-LUMO gap and the GCN-predicted gap of the generated molecules, quantified using MAE. Each metric is calculated as

\begin{equation}\label{eq:MAE}
  MAE = \frac{1}{N}\sum^{N}\vert y_{pred}-y_{true} \vert
\end{equation}

\begin{equation}\label{eq:RMSE}
  RMSE = \sqrt{\frac{1}{N}\sum^{N} (y_{pred}-y_{true})^{2}}
\end{equation}

\begin{equation}\label{eq:R2}
  R^2 = 1 - \frac{\sum^{N} (y_{true}-y_{pred})^{2}}{\sum^{N} (y_{true}-\bar{y})^{2}}
\end{equation}

\begin{equation}\label{eq:CRA}
  \overline{CRA} = \frac{1}{N} \sum_{n=1}^{N} \frac{1}{T_n} \sum_{t=1}^{T_n} \mathbf{1} (\hat{x}_{n,t} = x_{n,t})
\end{equation}

with $N$ the total number of test molecules and $T_n$ the number of characters per SMILES string.

\begin{equation}\label{eq:validity}
  Validity = \frac{1}{n_{trials}} \sum_{i=1}^{n_{trials}} \mathbf{1} (isValid(\hat{x}_{i}))
\end{equation}

\subsection{LightGBM}
\noindent
LightGBM is a gradient-boosted decision tree algorithm optimized for speed and high-dimensional data. It grows trees in a leaf-wise rather than level-wise manner, allowing it to capture complex, non-linear relationships more efficiently than many traditional boosting methods. This makes it well-suited for tabular molecular descriptor data, as it is robust to feature scaling, multicollinearity, and missing values.
\newline
\noindent
Hyperparameter optimization for the LightGBM model was carried out using Optuna.
The search space included parameters controlling model complexity and regularization:
the number of leaves (num\_leaves, 16--256), the learning rate (learning\_rate, $1\times10^{-3}$--0.1, log-scaled), feature subsampling (feature\_fraction, 0.7--1.0), sample bagging (bagging\_fraction, 0.7--1.0; bagging\_freq, 1--5), and the minimum number of samples per leaf (min\_data\_in\_leaf, 10--100). Other LightGBM settings, such as the boosting type (gbdt) and objective function (regression\_l1), were kept fixed throughout. Optuna selected the optimal hyperparameter configuration by minimizing the mean absolute error (MAE) obtained from the inner loop of the nested cross-validation, using a total of 50 trials.
\newline
\newline
\noindent
After nested cross-validation, the best-performing hyperparameters were selected (either by averaging over outer folds or by majority vote). The final LightGBM model was then trained on the entire dataset to produce predictions for comparison with the GCN model.

\subsection{GCN}
\noindent
The implementation of the GCN model was based on a tutorial about creating a simple Pytorch-based GCN\parencite{deshmukh2023building}. Some edits to the tutorial code were made, for example the addition of node features, the addition of an $R^2$ metric, and correcting a mistake with the standardization of the loss.
\newline
\newline
\noindent
Starting from the node and edge matrix representations of a graph, the GCN model first applied a graph convolution using multiple convolution layers. Each convolution layer gives a node information about its neighbors using the matrix multiplication shown in \Cref{fig:graph_conv_layer}. After the convolution layers $pooling$ is applied, which turns the 2-dimensional matrix into a 1-dimensional vector that contains the means of every column of the node matrix. This vector can then be passed to the neural network, which is a simple multilayer perceptron (MLP), a fully connected feedforward neural network.

\begin{figure}[H]
    \centering
    \includegraphics[width=0.7\textwidth]{img/graph_conv_layer.png}
    \caption{Graph convolution for an acetamide molecule\parencite{deshmukh2023building}.}
    \label{fig:graph_conv_layer}
\end{figure}

\noindent
Hyperparameter optimization for the GCN model was performed using a simple grid search, as this allowed easy parallelization of hyperparameter tuning. The parameters and their possible values included in the grid search are: batch\_size (64, 128, 256, 384), hidden\_nodes (64, 96, 128), n\_conv\_layers (1-5), n\_hidden\_layers (1-3), learning\_rate (0.001, 0.003, 0.005, 0.007, 0.01). Other parameters such as the maximum dimensions of the node and edge matrix, and the number of epochs were kept constant. After nested cross-validation, the same best-performing hyperparameters were obtained for each outer fold: batch\_size = 256, hidden\_nodes = 128, n\_conv\_layers = 4, n\_conv\_layers = 2, learning\_rate = 0.003, the number of epochs was kept constant at 50.

\subsection{cVAE}\label{section:cvae_method}
\noindent
The conditional variational autoencoder (cVAE) is chosen for its probabilistic nature, which enables sampling of new molecules with higher validity than a standard autoencoder \parencite{kingma_2019}. Our implementation is partially inspired by \parencite{bombarelli_2018, lim_2018}, except for the encoder. The model consists of a graph-based GCN encoder, a continuous conditioned latent space, and a GRU sequence decoder \parencite{yuan_2020, cho_2014}, combining the richer information from graph representations with the simplicity of SMILES decoding.
\newline
\newline
\noindent
Training maximizes the conditional ELBO (equivalently, minimizes the negative ELBO), composed of a categorical cross-entropy reconstruction loss and a Kullback-Leibler divergence regularizing the latent space\parencite{beckham_2023}:

\begin{equation}
  \mathcal{L}_{cVAE} = \mathbb{E} [\log \: p(x | z, y)] - \beta D_{KL} (q(z|x,y) \| \mathcal{N} (0, I))
\end{equation}
\noindent
where $q(z|x,y)$ is the encoder's approximate posterior, mapping a molecule and its HOMO-LUMO gap to a latent distribution, and $p(x|z,y)$ is the decoder's likelihood, giving the probability of reconstructing a SMILES sequence from a latent vector and conditioning value.
\newline
\newline
\noindent
Hyperparameters are optimized using Bayesian optimization with a TPE sampler \parencite{watanabe_2023}, although only a limited search is feasible, consisting of only 15 trials, due to the large parameter space and time constraints. For the same reason, five inner folds are used for cross-validation instead of ten, but given the large size of the QM9 dataset, this reduction is not expected to significantly affect the results. An overview of the selected optimal hyperparameters is provided in \cref{tab:cvae_params}.
\newline
\newline
\noindent
The optimal set of hyperparameters is used to generate new molecules using two approaches: random sampling of latent vectors from the latent space and perturbation of latent vectors corresponding to molecules in the dataset by adding noise drawn from a normal distribution scaled by a factor of 0.5. In both cases, the latent vectors are decoded while conditioning on target HOMO-LUMO gaps.
\newline
\newline
\noindent
For each experimental pathway, ten molecules with distinct target HOMO-LUMO gaps are selected. For each molecule, 1,000 generations are performed, and the average chemical validity is computed, as described in \cref{sec:evalmetrics}. For the valid generations of each molecule, the mean absolute error (MAE) is calculated by comparing the HOMO-LUMO gap predicted by the GCN model with the corresponding target gap, yielding a per-molecule MAE.

\section{Results}
% Here you describe the results you obtained. This is not just a big data dump where you put all the numbers into one big table. Try to point out patterns or striking things in your results. Think about how you can visualize the results so that a reader, who has not conducted your research themselves, can quickly understand what your results mean.
\subsection{LightGBM}
\label{sec:lightgbm}
For the performance of the model we first looked to the mean absolute error (MAE), root mean square error (RMSE), and $R^2$ between the predicted and true HOMO-LUMO gap which are calculated for each outer cross-validation fold. These results are shown in \cref{tab:LightGBM} and \cref{fig:LightGBM_R2}. We see that the values for the different metrics does not change much with each fold which is in line whit the chosen stratified splits. The mean MAE of 0.1418 eV is in line with the work of \textcite{Hasan2025}. 
\newline
\newline
\noindent

\begin{table}[H]
\caption{LightGBM : mean absolute error (MAE), root mean square error (RMSE), and $R^2$ between predicted and target HOMO-LUMO gaps per outer fold. Mean values across all five folds are also reported.}
\label{tab:LightGBM}
\centering
\small
\setlength{\tabcolsep}{16pt}   % column padding
\renewcommand{\arraystretch}{1.5} % row height
\begin{tabular}{|l|c|c|c|}
\hline
 & MAE (eV) & RMSE (eV) & $R^2$ \\ \hline
Outer Fold 1 & 0.1417 & 0.2160 & 0.9715\\ \hline
Outer Fold 2 & 0.1408 & 0.2122 & 0.9727 \\ \hline
Outer Fold 3 & 0.1411 & 0.2200 & 0.9705 \\ \hline
Outer Fold 4 & 0.1430 & 0.2183 & 0.9709 \\ \hline
Outer Fold 5 & 0.1423 & 0.2227 & 0.9696 \\ \hline
Mean & 0.14 & 0.22 & 0.97 \\ \hline
\end{tabular}
\end{table}

\begin{figure}[H]
    \centering
    \includegraphics[width=0.4\textwidth]{img/LightGBM_R2.png}
    \caption{Predicted vs true HOMO-LUMO gap for all 5 outer folds together of the LigthGBM model.}
    \label{fig:LightGBM_R2}
\end{figure}

The feature importance stability plot (\cref{fig:LightGBM_features}) summarizes the mean and standard deviation of LightGBM feature importances across the five outer cross-validation folds for the top 20 descriptors. Points indicate the average importance of each descriptor, while error bars reflect variability across folds, providing insight into both feature relevance and robustness. Several descriptors consistently dominate the model, most notably MinAbsEStateIndex, BCUT2D$_{\text{MWLOW}}$, and BCUT2D$_{\text{MRHI}}$, which exhibit high mean importance with low variability. This indicates that these features are both highly informative and stable predictors of the HOMO--LUMO gap.
\newline
\newline
\noindent
Overall, feature importance shows a gradual decay, with approximately the top ten descriptors accounting for most of the predictive power. The low standard deviations observed for the most important features suggest that the model relies on a robust core of chemically meaningful descriptors that remain consistent across different training subsets.These results highlight the relevance of descriptors capturing electronic and topological properties, such as BCUT metrics and EState indices, for accurate HOMO-LUMO gap prediction, while lower-importance features appear to contribute only marginally to overall performance.
\begin{figure}[H]
    \centering
    \includegraphics[width=0.4\textwidth]{img/LightGBM_features.png}
    \caption{The top 20 most important features across the 5 outer folds where the error bar reflect variability across folds.}
    \label{fig:LightGBM_features}
\end{figure}
\subsection{GCN}

Just as for the LightGBM model, the mean absolute error (MAE), root mean square error (RMSE), and $R^2$ between the predicted and true HOMO-LUMO gap were determined for the GCN model. All evaluation metrics of the GCN model (\Cref{tab:GCN}) have slightly higher values than those of the LightGBM model (\Cref{tab:LightGBM}). The consistent appearance of the same group of outliers across all outer folds (upper left half of each $R^2$ plot in \Cref{fig:GCN_R2}) indicates that the stratified outer cross-validation split was balanced.

\begin{table}[H]
\caption{GCN: Mean absolute error (MAE), root mean square error (RMSE), and $R^2$ between predicted and target HOMO-LUMO gaps per outer fold. Mean values across all five folds are also reported.}
\label{tab:GCN}
\centering
\small
\setlength{\tabcolsep}{16pt}   % column padding
\renewcommand{\arraystretch}{1.5} % row height
\begin{tabular}{|l|c|c|c|}
\hline
 & MAE (eV) & RMSE (eV) & $R^2$ \\ \hline
Outer Fold 1 & 0.1677 & 0.2403 & 0.9646 \\ \hline
Outer Fold 2 & 0.1585 & 0.2280 & 0.9682 \\ \hline
Outer Fold 3 & 0.1567 & 0.2255 & 0.9690 \\ \hline
Outer Fold 4 & 0.1586 & 0.2231 & 0.9697 \\ \hline
Outer Fold 5 & 0.1587 & 0.2299 & 0.9679 \\ \hline
Mean & 0.16 & 0.23 & 0.97 \\ \hline
\end{tabular}
\end{table}

\begin{figure}[H]
    \centering
    \includegraphics[width=0.9\textwidth]{img/GCN_R2.png}
    \caption{Predicted vs true HOMO-LUMO gap for all 5 outer folds of the GCN model.}
    \label{fig:GCN_R2}
\end{figure}

\subsection{cVAE}

The reconstruction performance of the cVAE model, quantified by the average character-wise reconstruction accuracy $\overline{CRA}$, across the five outer cross-validation folds is reported in \cref{tab:CRA}. The model achieves an overall mean reconstruction accuracy of 0.852, indicating that it is generally capable of accurately reconstructing individual molecules. Notably, outer folds three and five exhibit substantially higher reconstruction accuracies than the remaining folds. This improvement can be attributed to more favorable hyperparameter configurations identified during their respective tuning procedures (\cref{tab:cvae_params}), in particular a higher learning rate, a larger GRU hidden dimension, and a lower $\beta$ parameter.

\begin{table}[H]
\centering
\caption{The average character-wise reconstruction accuracy, as calculated by \cref{eq:CRA}, of the test set per outer fold of the cross-validation.}
\label{tab:CRA}
\begin{tabular}{|c|c|}
\hline
 & $\overline{CRA}$ \\ \hline
Outer fold 1 & 0.842 \\ \hline
Outer fold 2 & 0.840 \\ \hline
Outer fold 3 & 0.871 \\ \hline
Outer fold 4 & 0.846 \\ \hline
Outer fold 5 & 0.863 \\ \hline
Mean & 0.852 \\ \hline
\end{tabular}
\end{table}

Beyond reconstruction performance, the primary metric of interest is the generative capability of the cVAE. Using the best-performing trained model (outer fold 3), the chemical validity of molecules generated from both randomly sampled latent vectors and perturbed latent vectors is evaluated (\cref{fig:validity} and \cref{tab:cvae_samples}). Across the ten sampled target molecules in each experimental pathway, the average validity is low and comparable between the two sampling strategies, with no clear advantage observed for either method.
\newline
\newline
For each target molecule, the mean absolute error (MAE) between the HOMO-LUMO gaps predicted by the GCN model and the corresponding target gaps used to condition the cVAE (\cref{section:cvae_method}) is computed and reported in \cref{fig:cvae_maes,tab:cvae_samples}. It is important to note that this MAE is calculated per molecule and depends on the number of valid generations, as only valid molecules contribute to the error calculation. Consequently, the validity indirectly influences the MAE. For both sampling methods, the perturbed and random latent sampling methods yield MAE values that are considerably larger than the errors reported for the standalone GCN model, meaning there is a significant difference between the target and generated HOMO-LUMO gap.

\begin{figure}[H]
  \centering
  \begin{subfigure}{0.4\textwidth}
    \centering
    \includegraphics[width=\linewidth]{img/avg_validity.png}
    \caption{The average validity for both 10 randomly sampled latent vectors and 10 perturbed latent vectors with their respective 95\% confidence intervals.}
    \label{fig:validity}
  \end{subfigure}
  \hfill
  \begin{subfigure}{0.4\textwidth}
    \centering
    \includegraphics[width=\linewidth]{img/cvae_maes.png}
    \caption{The MAEs of the valid generated molecules for each of the 10 random and 10 perturbed samples.}
    \label{fig:cvae_maes}
  \end{subfigure}
  \caption{A visual representation of the data from \cref{tab:cvae_samples}.}
\end{figure}

To further explore the perturbed latent space vectors, we visualize the generated chemical structures to assess how effectively the cVAE can produce molecules with the same HOMO-LUMO gap while exhibiting slight structural variations. \Cref{fig:genmol1,fig:genmol5,fig:genmol10} present three starting molecules alongside four of their respective perturbed generated molecules.


\section{Discussion}
% Provide an interpretation of your results. What do they mean in the context of your research question? What do you notice here compared to other papers you've read? Perhaps a particular method is not as superior to all the others after all as you originally expected, or maybe you had an idea for a technique that seemed good in theory but doesn't work as well in practice, or \dots


\subsection{LightGBM and GCN have similar performances}
\noindent
The LightGBM model achieves a mean absolute error (MAE) of 0.14 eV and an $R^2$ of 0.97 for HOMO-LUMO gap prediction across all cross-validation folds (\Cref{tab:LightGBM}). The consistency of these metrics across folds indicates that the stratified cross-validation splits are well balanced. Our results can be directly compared to those of \textcite{Hasan2025}, as they employ the same descriptor-based approach using RDKit-generated molecular descriptors as input to their machine learning models. In their study, a LightGBM model achieved a MAE of 0.1675 eV, which is higher than the error obtained in the present work. One likely contributing factor is the hyperparameter optimization strategy: while \textcite{Hasan2025} rely on a grid search, we employ Optuna, a Bayesian optimization framework that enables more efficient exploration of the hyperparameter space.
\newline
\newline
\noindent
Feature importance analysis further reveals that approximately 50 descriptors contribute meaningfully to the prediction of the HOMO-LUMO gap, with around 20 descriptors exhibiting particularly strong influence, while the remaining features provide only minor refinements to the predictions. The most influential descriptors predominantly capture topological and electronic properties of the molecules, whereas atom-level features are less important. This observation is expected, as variations in atomic composition are implicitly reflected in global electronic and topological descriptors.
\newline
\newline
\noindent
The GCN model achieves a mean MAE of approximately 0.16 eV for HOMO–LUMO gap prediction, with a consistently high $R^2$ of about 0.97 across all cross-validation folds (\Cref{tab:GCN}), indicating a balanced split. This performance improves upon a very recent GCN result reported for the PCQM4Mv2 dataset, where a MAE of around 0.20 eV was observed\parencite{hu2021ogblsc}, and is competitive with other GCN based approaches reporting MAEs of 0.14 eV\parencite{choi2022scalabletraininggraphconvolutional} and 0.12 eV\parencite{Therrien2025}. Although the present model does not surpass the best-performing architectures in the literature, the relatively small performance gap suggests that a standard 2D GCN can capture most of the relevant structure–property relationships governing the HOMO–LUMO gap. Overall, these results reinforce the effectiveness of graph convolutional networks for molecular electronic property prediction while highlighting that further gains likely require more expressive models or additional structural information.
\newline
\newline
\noindent
Although prior work has shown that deep-learning models do not consistently outperform non-deep models for molecular property prediction\parencite{NEURIPS2023_cc83e973}, our results show that the GCN attains comparable accuracy to that of the LightGBM model. While the GCN model learns chemically meaningful structural relationships through a graph, the LightGBM model uses descriptors of the molecule to learn mappings between these features and the target property. Rather than clearly favoring one approach over the other, our findings suggest that deep-learning and traditional models address the the prediction of HOMO-LUMO gap from complementary perspectives while achieving similar overall performance. A more comprehensive comparison across a wider range of models and datasets, similar to the analysis in \textcite{NEURIPS2023_cc83e973} but focused specifically on HOMO–LUMO gap prediction, would help clarify the relative advantages of traditional versus deep-learning approaches for this specific task.

\subsection{Molecular Generation with cVAE}

The mean character-wise reconstruction accuracy ($\overline{CRA} = 0.852$, \cref{tab:CRA}) demonstrates that the model can recover the original molecular structures in most cases. While reconstruction accuracies for cVAEs on the exact same dataset are not available, \textcite{blaschke_2018} report 0.862 with teacher forcing (0.963 without) for a standard VAE, and \textcite{mollaysa_2024} achieved 0.961 with a Transformer VAE. These results show that our cVAE performs just below literature standard, however, our limited hyperparameter tuning suggests further improvements are possible.
\newline
\newline
Despite this reconstruction performance, the chemical validity of both the randomly sampled and perturbed generated molecules is low and comparable (\cref{fig:validity}), which is surprising given prior work showing that local latent sampling often increases validity \parencite{lim_2018}. A likely contributing factor to the overall low validity is the high teacher forcing ratio employed during training (\cref{tab:cvae_params}), which improves reconstruction by conditioning the decoder on ground-truth tokens but reduces its robustness during free generation. Consequently, the model learns to reconstruct known sequences well but struggles to generate valid novel molecules when sampled freely.
\newline
\newline
In addition, the model struggles to generate molecules closely resembling structures from the dataset (\cref{fig:genmol1,fig:genmol5,fig:genmol10}). Although the model captures certain chemical motifs, such as the ring structures observed in \cref{fig:genmol10}, these patterns are not consistently preserved across the generated samples. This contrasts \textcite{lim_2018} and \textcite{bombarelli_2018}, who reported structurally similar generated molecules. The lack of structural preservation observed here suggests that the learned latent space does not encode fine-grained molecular structure and a deeper analysis of the quality of the latent space is required to find the origin of this issue.
\newline
\newline
Finally, the large mean absolute errors observed for the HOMO-LUMO gaps of generated molecules further indicate limited conditional control (\cref{tab:cvae_samples}). The errors are substantially larger than those reported in literature \parencite{bombarelli_2018}, highlighting that the conditional encoding is insufficient to reliably enforce target electronic properties, especially given low chemical validity.

\section{\textcolor{purple}{Conclusion}}

% Here, you summarize your research and the conclusions you draw from it. Also mention what shortcomings you see in your research. Perhaps there are still certain experiments you would have liked to have done, but didn't have time to do? Think about what steps someone could take to move your research forward.
\noindent
From a predictive modeling perspective, the LightGBM approach demonstrated strong and reliable performance for HOMO--LUMO gap estimation. By leveraging RDKit-based molecular descriptors and Bayesian hyperparameter optimization with Optuna, the model achieved a mean absolute error of 0.14~eV and an $R^2$ of 0.97, which is competitive with, and in some cases superior to, recent literature results \parencite{Hasan2025}. The stability of the performance across cross-validation folds and the consistency of feature importance rankings indicate that the model captures robust structure-property relationships rather than overfitting to specific subsets of the data. Furthermore, the dominance of topological and electronic descriptors among the most influential features highlights the physical relevance of the learned mappings. These results confirm that well-tuned, descriptor-based ensemble models such as LightGBM remain highly effective baselines for molecular electronic property prediction, particularly when large, high-quality datasets are available.
\newline
\newline
In this work, a conditional variational autoencoder was developed to generate molecular SMILES conditioned on the HOMO-LUMO gap. While the model achieved a reasonable reconstruction accuracy of 0.852, the generated molecules exhibited low chemical validity, limited structural similarity to source molecules and large deviations from the target HOMO-LUMO gaps. Further improvements should focus on increasing the reconstruction accuracy through more extensive hyperparameter tuning, enhancing generative robustness by, among other factors, more careful handling of the teacher forcing ratio, and systematically assessing the quality of the learned latent space using methods proposed in prior work \parencite{bombarelli_2018,lim_2018}. Future research should also explore improving the encoder architecture and replacing the SMILES-based decoder with a graph-based decoder, which could explicitly address graph isomorphism and prevent the generation of strings that do not correspond to valid molecular graphs \parencite{bombarelli_2018}.

\section{Disclaimers}
% If you used ChatGPT or other LLMs to do your research and/or write your text, please mention that here. If you used such tools for your research itself, please also mention the prompts you used. Don't be afraid that we will give you a lower grade because you used an LLM. We won't.


\appendix

\section{cVAE Hyperparameters}

\begin{table}[H]
\centering
\caption{The optimized cVAE hyperparameters for each outer fold achieved after running 15 trials in a Bayesian optimization algorithm using a TPE sampler. The number of epochs and batch size were fixed to narrow the searching space of the Bayesian algorithm.}
\label{tab:cvae_params}
\begin{tabular}{|c|c|c|c|c|c|}
\hline
 & fold 1 & fold 2 & fold 3 & fold 4 & fold 5 \\ \hline
learning rate & 0.00133 & 0.00133 & 0.00548 & 0.00133 & 0.00548 \\ \hline
latent dimension & 62 & 62 & 50 & 62 & 50 \\ \hline
$\#$hidden nodes & 79 & 79 & 66 & 79 & 66 \\ \hline
GRU dimension & 42 & 42 & 64 & 42 & 64 \\ \hline
$\#$convolution layers & 1 & 1 & 1 & 1 & 1 \\ \hline
$\#$hidden layers & 1 & 1 & 2 & 1 & 2 \\ \hline
$\#$GRU layers & 1 & 1 & 1 & 1 & 1 \\ \hline
$\#$FC layers & 3 & 3 & 2 & 3 & 2 \\ \hline
embedding dimension & 18 & 18 & 14 & 18 & 14 \\ \hline
teacher forcing ratio & 0.876 & 0.876 & 0.888 & 0.876 & 0.888 \\ \hline
$\beta$ & 4.62 & 4.62 & 1.88 & 4.62 & 1.88 \\ \hline
batch size & 1000 & 1000 & 1000 & 1000 & 1000 \\ \hline
$\#$epochs & 15 & 15 & 15 & 15 & 15 \\ \hline
\end{tabular}
\end{table}

\section{cVAE results}

\begin{table}[H]
\centering
\caption{The validity and MAE of the random samples and perturbed samples, obtained as described in \cref{section:cvae_method}.}
\label{tab:cvae_samples}
\begin{tabular}{|c|cc|cc|}
\hline
 & \multicolumn{2}{c|}{Random samples} & \multicolumn{2}{c|}{Perturbed samples} \\ \hline
Molecule & \multicolumn{1}{c|}{Validity} & \begin{tabular}[c]{@{}c@{}}gap MAE\\ (eV)\end{tabular} & \multicolumn{1}{c|}{Validity} & \begin{tabular}[c]{@{}c@{}}gap MAE\\ (eV)\end{tabular} \\ \hline
1 & \multicolumn{1}{c|}{0.066} & 0.605 & \multicolumn{1}{c|}{0.042} & 0.918 \\ \hline
2 & \multicolumn{1}{c|}{0.054} & 0.543 & \multicolumn{1}{c|}{0.018} & 0.926 \\ \hline
3 & \multicolumn{1}{c|}{0.048} & 0.886 & \multicolumn{1}{c|}{0.045} & 0.620 \\ \hline
4 & \multicolumn{1}{c|}{0.011} & 1.135 & \multicolumn{1}{c|}{0.045} & 0.748 \\ \hline
5 & \multicolumn{1}{c|}{0.066} & 0.726 & \multicolumn{1}{c|}{0.049} & 0.796 \\ \hline
6 & \multicolumn{1}{c|}{0.018} & 0.992 & \multicolumn{1}{c|}{0.034} & 0.609 \\ \hline
7 & \multicolumn{1}{c|}{0.044} & 0.467 & \multicolumn{1}{c|}{0.055} & 0.806 \\ \hline
8 & \multicolumn{1}{c|}{0.036} & 0.817 & \multicolumn{1}{c|}{0.061} & 0.707 \\ \hline
9 & \multicolumn{1}{c|}{0.048} & 0.904 & \multicolumn{1}{c|}{0.026} & 0.890 \\ \hline
10 & \multicolumn{1}{c|}{0.061} & 0.850 & \multicolumn{1}{c|}{0.042} & 0.369 \\ \hline
\end{tabular}
\end{table}

\begin{figure}[H]
  \centering
  \includegraphics[scale=0.4]{img/genmol1.png}
  \caption{Starting molecule 1 from \cref{tab:cvae_samples} and four of its perturbed generated molecules.}
  \label{fig:genmol1}
\end{figure}

\begin{figure}[H]
  \centering
  \includegraphics[scale=0.4]{img/genmol5.png}
  \caption{Starting molecule 5 from \cref{tab:cvae_samples} and four of its perturbed generated molecules.}
  \label{fig:genmol5}
\end{figure}

\begin{figure}[H]
  \centering
  \includegraphics[scale=0.4]{img/genmol10.png}
  \caption{Starting molecule 10 from \cref{tab:cvae_samples} and four of its perturbed generated molecules.}
  \label{fig:genmol10}
\end{figure}

\printbibliography
\end{document}